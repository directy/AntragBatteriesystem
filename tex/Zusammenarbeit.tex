
\chapter{Zusammenarbeit mit Dritten}
\label{sec:ZusammenarbeitMitDritten}


\begin{enumerate}


\item	Das Biosphärenreservat Spreewald sichert finanzielle, informelle sowie organisatorische Unterstützung zu.
Es soll ein Auftrag für ein Gutachten über die Einsatzmöglichkeiten der Elektromobilität in Spreewaldkähnen an uns erfolgen.

\item	Der Wasser- und Bodenverband Oberland Calau Hat als Pflege- und Instandhaltungseinrichtung für den Spreewald zahlreiche Kähne und professionelles Personal sowie bereits langjährige Erfahrung mit Elektroantrieben.

\item	Das Landesamt für Bauen und Verkehr Brandenburg Möchte mit uns hinsichtlich der Gestaltung der Vorschriften für die technische Umsetzung der Umrüstung der Wasserfahrzeuge zusammenarbeiten und zählt auf unsere wissenschaftliche Kopetenz


\item	Das Autocenter Kamenz, vertreten durch den Geschäftsführer Herrn Böttcher, der ebenfalls jahrzehntelange Erfahrung mit Elektroantrieben in Wasserfahrzeugen hat, unterstützt technisch, durch zahlreiche wertvolle Informationen sowie durch die Bereitstellung eines privaten und geeigneten Sees zu Testzwecken.
\item Schlodarik, Leider Wasser und Bodenverband Calau
\item BudichPool GmbH
	
\end{enumerate}

%\subsection{Kooperationspartner und Arbeitsteilung}
%
%Als Kooperationspartner zum Thema EKahn zeichnet die Firma Budichpool GmbH verantwortlich. Sie ist ein regional im Spreewald ansässiges Unternehmen mit langjähriger Erfahrung im Bereich Metallverarbeitung, Konstruktionstechnik, Planung und PVC-Verarbeitung. Die nötigen Räumlichkeiten, Maschinen und Werkzeuge sowie Personal mit entsprechenden fachlichen Qualifikationen sind vorhanden und für das Projekt eingeplant. Ansprechpartnerist  Herr Burkhard Budich.
%
%Das Unternehmen Budichpool GmbH wurde 1990 gegründet - als GbR  (Burkhard und Frank Budich) und umfasste Folienschweißtechnik, Reifenservice sowie Schwimmbeckenbau. Im Jahr 1994 erfolgte die Gründung einer GmbH (Schwimmbecken- und Freizeitanlagen Budich GmbH).
%Das Unternehmen hat seinen Sitz in 15907 Lübben, an der Feuerwache 2 im Landkreis Dahme Spreewald (LDS). 
%In den vergangenen Jahren wurden zahlreiche Neuentwicklungen im Schwimmbadbereich durchgeführt, wie z. B. Neuentwicklung von Formsteinbecken und Segmentbecken. Letzteres erfolgte mit einer Gebrauchsmustereintragung.  Die Produkte wurden überwiegend in Brandenburg, Sachsen und Berlin vertrieben.
%
%Innovative Aspekte:
%In Zukunft wird sich die Firma mit dem Umbau von Spreewaldkähnen beschäftigen, dies beinhaltet sowohl die Umrüstung von Verbrennungskraftmaschinen auf Elektroantriebe als auch die Entwicklung neuer innovativer Überdachungskonzepte. Damit soll der Einsatz der Spreewaldkähne, welche maßgeblicher für den Tourismus im Spreewald verantwortlich sind, auch bei schlechten Witterungsverhältnissen möglich sein. Es wird davon Ausgegangen, dass der derzeitige Einsatz (Mai-September) um 2-3 Monate erhöht werden kann. Die Firma hat daher größtes Interesse sich an diesem Projekt zu beteiligen um ein Schlüssiges Gesamtkonzept (elektro Kahn + Überdachung) anbieten zu können.

