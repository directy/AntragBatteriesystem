
\chapter{Stand der Wissenschaft und Technik}
\label{sec:StandDerWissenschaftUndTechnik}


\section{Light Electrical Vehicle (LEV)}
\label{sec:LightElectricalVehicleLEV}


\section{Ladeinfastruktur}
\label{sec:Ladeinfastruktur}


\subsection{Autarke Ladesäulen}
\label{sec:AutarkeLadesäulen}


\subsection{Möglichkeiten der Aufladung}
\label{sec:MöglichkeitenDerAufladung}


\section{Solarmodule}
\label{sec:Solarmodule}


\section{Batteriemodule}
\label{sec:Batteriemodule}


Aus dieser Aufgabenstellung ergeben sich folgende Anforderungen an das modulare
\begin{itemize}

\item \begin{itemize}Allgemeine Anforderungen
			\item Spannung
			\item Kapazität
			\item Akkumulatortyp
			\item Sicherheit
			\item Schnittstellen
			\item Informationen
			\item Kommunikation
			\end{itemize}
\item \begin{itemize}Konstruktive Anforderungen
			\item Form
			\item  Gewicht
			\item  Baugröße
			\item  Mechanische Festigkeit
			\item  Bedienbarkeit
			\item  Äußeres Erscheinungsbild
			\item  Aufnahmesystem
			\end{itemize}
\end{itemize}





\section{Bisherige Arbeiten des Antragstellers}
\label{sec:BisherigeArbeitenDesAntragstellers}

Einer der Ausgangspunkte für die Entwicklung eines modularen Akkumulatorsystems ist das Projekt "`Untersuchung über die Einsatzmöglichkeiten von Elektromobilität in Spreewaldkähnen"'. In diesem Projekt werden die Einsatzmöglichkeiten der
elektrischen Mobilität in Wasserfahrzeugen am Beispiel der Spreewaldkähne im
Biosphärenreservat Spreewald untersucht.

% CO2-Thema obwohl ich nicht weiß ob man das erwähnen sollte
% Saxmobility
% optimierte Fahrzeuglängsführung


\section{Antrag Stettler}

