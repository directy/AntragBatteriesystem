
\chapter{Verwertungsplan}
\label{sec:Verwertungsplan}


\section{Wirtschafliche Erfolgsasussichten}
\label{sec:WirtschaflicheErfolgsasussichten}

\subsubsection{Verwertungsperspektive Firma BudichPool }
Die Firma Budichpool als unser Kooperationspartner erhofft sich durch die Schaffung des Gesamtsystems zur elektrischen Fortbewegung in Wasserfahrzeugen Aufträge hinsichtlich der technischen Umrüstung der Boote. Daher rührt auch die Bereitschaft zur Zusammenarbeit und der finanziellen Unterstützung. Nach Ablauf des dreijährigen Entwicklungszeitraumes  soll eine Firmenneugründung stehen, die sich nach der Entwicklung des geplanten Batteriesystems mit deren Herstellung und Vermarktung befasst. Dieses Unternehmen soll zusätzlich auch eine Ladeinfrastruktur in Wassersportgebieten sowie einen Verleihservice auf den Markt bringen. 
In Verbindung mit den entsprechenden Herstellern sollen auch die Einsatzmöglichkeiten des Batteriesystems in anderen Elektrofahrzeugen sowie möglichen weiteren Anwendungen ausgelotet werden.  

Es ist noch festzuhalten, dass dieses Projekt für alle Beteiligten ein gewisses technisch-wissenschaftliches und wirtschaftliches Risiko birgt. 

\subsubsection{Verwertungsperspektive Tourismusverband Lübben }
\subsubsection{Anreizeffekt}

In der Zusammenarbeit zwischen unserer Hochschule, einem Wirtschaftsunternehmen und anderen teils behördlichen Partnern  sehen wir die optimale Konstillation für die zeitnahe Erreichung der gesteckten Ziele. Da die finanziellen Mittel an den Hochschulen jedoch oft nur eine unbefriedigende Entwicklungsgeschwindigkeit ermöglichen, sind wir uns absolut sicher, mit der Unterstützung durch die Förderungmaßnahme "ELEKTROmobilität: Positionierung der neuen WERtschöpfungskette (ELEKTRO POWER)" die Ziele deutlich schneller und qualitativ hochwertiger zu erreichen um damit auf dem deutschen Markt Fuss zu fassen. Auch unser Projektpartner sieht sich im Stande bei einer derartigen Beihilfe seine eingebrachten Mittel aufzustocken. 
\subsubsection{Biospährenrervat Spreewald}

\subsubsection{Landesamt für Bauen und Verkehr }


\subsection{Warum LEVs Möglichkeiten und Grenzen?}
\label{sec:WarumLEVsMöglichkeitenUndGrenzen}


\subsection{Solarboote Unterstützung in der FahrgastSchiiffahrt?}
\label{sec:SolarbooteUnterstützungInDerFahrgastSchiiffahrt}

\section{Ausgründung einer Firma}
\label{sec:AusgründungEinerFirma}




\section{Wissenschaftliche und/oder technische Erfolgsaussichten}
\label{sec:WissenschaftlicheUndOderTechnischeErfolgsaussichten}

Überprüfung auf Patentwürdigkeit!

\section{Wissenstransfer}
\label{sec:Wissenstransfer}

Ergebnisse werden der öffentlichkeit über ein WIKI bzw. Internetpräs bekannt gegeben.

Öffentlichkeitswirksam sehr hoch
Gut für die Emobilität

