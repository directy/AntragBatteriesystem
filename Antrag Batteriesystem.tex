



\documentclass[12pt,a4paper]{report}


\usepackage{german}      % Deutsche TeX-Eigenheiten
%\usepackage[utf8]{inputenc} 
\usepackage[latin1,utf8]{inputenc}
\usepackage[T1]{fontenc} 


\usepackage{makeidx}
\makeindex            % damit eine Indexdatei angelegt wird


%%
%% AMS-LaTeX Pakete:
%%
\usepackage{amsmath}  % allgemeine Mathe-Erweiterungen
\usepackage{amssymb}  % Symbole und Schriftarten
\usepackage{amsthm}   % erweiterte Theorem-Umgebungen


%%
%% XY-Pic für Diagramme etc
%%
%\usepackage[all]{xy}      % Das Paket mit allem, was man so braucht
%\UseComputerModernTips    % Pfeilspitzen wie im normalen Mathe-Modus
%\CompileMatrices          % Damit geht es etwas schneller.


\usepackage{mathrsfs}  % gibt den Befehl "\mathscr{}" für schöne
                       % Mathe-Skript-Buchstaben 





%%
%% Für den Titel:
%%
%\author{G. Harder}
%\title{}
%\date{}   % Standard: Datum der Kompilierung ("\today").


%%%--------------------------%%%
%%%  Gleich geht es los...   %%%
%%%--------------------------%%%
\begin{document}
%%
%% Titel:
%%
%% * Entweder Standardtitel mit obigen "\title" und "\author"
%\maketitle
%% * oder alternativ eine frei formatierte Titelseite:
%\begin{titlepage}
%\end{titlepage}

%%
%% Zusammenfassung:
%%
%\begin{abstract}
%\end{abstract}

%%
%% Inhaltsverzeichnis:
%%
%\tableofcontents

%%%-------------------------------%%%
%%%  Jetzt geht es richtig los:   %%%
%%%-------------------------------%%%

Only a draft  version !

\chapter{Ziele des Vorhabens}
\label{sec:ZieleDesVorhabens}
y
% Themen
% Schaffung eines Batteriesystems
% Simulation/ReichweiteBerchnung/Onlinetool
% /SolarLadeinfastruktur

%Kurze Einleitung kann eventuell auch woanders hin oder als Fazit
Die Präsens von Light Electric Vehicle (LEVs) ist in Deutschland, im Vergleich zu anderen Ländern derzeit noch als eher gering einzustufen. Es kann aber davon ausgegangen werden, dass in naher Zukunft gerade dieses Segment von besondere Bedeutung sein wird. 

Ein Blick auf den asiatischen Raum, besonders in den Ballungszentren wie Shanghai oder Chongqing , zeigt, dass aufgrund der hier vorherrschenden mangelnde Platzressourcen überwiegend auf die kostengünstigen LEVs zurückgegriffen wird.(Rollerzahlen/Fußnote)
Ein solches Fahrzeug ist besonders im innerstädtischen Bereich für Kurzstrecken bestens geeignet.

Davon ausgehend, dass auch in Deutschland der urbane Bauraum eine immer knapper werdende Ressource darstellt und somit unter anderen Stellplätze für PKWs eine Mangelware oder zumindest zu einem  kostspieligen Unterfangen wird, wird die Zukunft auch hier zu einem Paradigmenwechsel führen und somit in einem erhöhten Interesse an LEVs münden.

%Marktdurchdingung/Marktanalysen
Bestätigt werden kann diese These durch Marktanalysen von Pedelecs, welche einen typischer Vertreter der LEVs darstellen \footnote{ genaue Übersetzung elektrisch unterstützter Fahrradantrieb}. 
%Marktsegment
%Marktanteil welche einen  vergleichsweise kleinem Marktsegment der LEVs
Zieht man hierfür, als Beispiel die Verkaufszahlen  als Berechnungsgrundlage heran, kann man erkennen, dass der im Verhältnis zum großen Marktsegment der LEVs als eher klein einzustufende Markt der Pedelecs bereits in den letzten Jahren eine Steigerung von bis zu 100000\% erfahren hat. Wobei die Tendenz als weiter steigend eingestuft werden kann.Quelle:Neupert.
Es ist somit mit einer starken Marktdurchdringung der LEVs im Segment der elektrischen Fortbewegung zu rechnen.

Der Fokus der Elektromobilität sollte demzufolge nicht einzig und alleine auf dem Gebiet der eklektischen PKWs liegen sondern auch den erhöhten Forschungsbedarf im Sektor der LEVs verfolgen.

Ziel dieses  Vorhabens ist die Entwicklung eines intelligenten, mannigfaltig einsetzbaren, ortsveränderlichen Energiespeichersystems zur Traktion von Light Electrical Vehicle (LEVs).
Es soll sich hierbei um ein intelligentes Batteriesystem handeln, welches sich je nach aktueller Anforderung und individuellen Eigenschaften (Alter, wie oft geladen and so on,..) optimal an die jeweilige Betriebssituation anpasst.
Besonders die vielfältigen Einsatzmöglichkeiten, sollen dieses System auszeichnen.

Auf dieser Grundlage, soll der verwendete chemische Energiespeicher durch eine Maximierung des Nutzungsfaktors in seiner Rentabilität um ein vielfaches gesteigert werden.

% Ausnutzungsfaktor
% wo liegt das Problem?
\section{Bekannte Probleme der Elektromobilität}

Bekannterweise verursacht der Traktionsspeicher einen Großteil der Kosten elektrisch betriebener Fahrzeuge.

Hochrechnungen im Automobilsektor haben ergeben das der installierte Energiespeicher in Elektrofahrzeugen die meiste Zeit ungenutzt bleibt ( 89 \% ). Diese Beobachtungen können durchaus auch auf andere elektrisch betriebene Fahrzeuge übertragen werden. So findet die verbaute Leistung in einem Fahrzeug meist nur im Fahrbetrieb ihren Verwendungszweck.

Zur Beurteilung wie effizient ein Batteriesystem ausgelegt wurde, kann der Kosten-/Nutzen-Faktor (CBF \footnote{cost benefit factor}) herangezogen werden.

Um diesen Faktor zu verbessern, können zum einen die Kosten des Energiespeichers vermindert (Herstellung)  oder der eigentliche Nutzen gesteigert werden.
Da die Kosten moderner Batterietechnologien in absehbarer Zukunft aber nicht zu einem akzeptablen CBF führen werden müssen andere Ansätze gefunden werden. 

Beobachtungen in der Automobilindustrie (im Bereich der Elektromobilität) zeigen Intentionen den CBF mit Hilfe von Smart Grid-Ansätzen \footenot{Beschreibung, Interesse durchaus auch von Autoherstellern nicht nur Energieversorger} zu verbessern.

Hierbei soll die Batterie zusätzlich als Zwischenspeicher für Regenerative Energien genutzt werden.

Die eher geringen installierten Leistungen bei LEVS ,sind für solche Zwecke aber nicht sinnvoll. 

Der Grundgedanke, den verbauten Energiespeicher so gut wie möglich auszunutzen soll hier jedoch aufgegriffen werden.
Da das zu entwickelnde System flexibel austauschbar sein soll, ist es möglich mit einem Energiespeicher mehre verschieden Endgeräte zu betreiben. So ist es z.B. denkbar, dass im privaten Haushalt mehrere  LEVs zur Verfügung stehen und diese je nach Bedürfnis mit dem Energiespeicher ausgerüstet werden. (Ein Energiespeicher für mehrere Fahrzeuge)

Die Steigerung des CBFs in LEVs soll eines der Primärziel dieses Forschungsvorhabens sein.

Durch intelligentes Management (Verringerung der mitgeführte Kapazität,..) soll die Nutzung optimiert und zum Anderen der Ausnutzungsgrad gesteigert werden.

Die folgenden Anforderungen an ein Batteriesystem werden angestrebt:
\begin{enumerate}
	\item Sicherheit, 
	\item Effizienz, 
	\item Benutzerfreundlichkeit, 
	\item Flexibilität und 
	\item Langlebigkeit.
\end{enumerate}

\section{geplanter Einbauort}
% Modulare Bauweise
Das zu entwickelnde Energiespeichersystem soll seinen Einsatz primär im Elektromobilitätssegment der Light Electric Vehicles finden.

Dieser Bereich  deckt laut Definition alle elektrischen Fahrzeuge die mit einer Spannung kleiner 60 V arbeiten und nicht in die Kategorie der KFZs eingeordnet werden können (z.B. Segway, Quads ,Elektroroller ,Fahrräder,Elektroboote) ab. 

Zusätzlich zum eigentlichen Nutzen des Energiespeichers als Traktionsspeicher, ist es denkbar weitere Verwendungszwecke zu generieren. (transportable Musikanlage / Kühlmöglichkeiten / einsetzbar in LEV-Roller)

%Damit ist eine Steigerung der Nutzungsperioden des Energiespeichers zu erwarten.

Der Einsatz von intelligenten Endverbrauchern\footenote{Ebenfalls Gegenstand dieses Forschungsvorhabens} in Kombination mit einem intelligenten Energiespeicher kann den Verbrauch und damit den Kosten/Nutzen-Faktor abermals steigern.


\section{Ladetechnik und Infrastruktur}

Die Verwendung eines intelligenten und genormten Energiespeichers haben auch Auswirkung auf die hierfür erforderlichen Ladeeinrichtung. So muss der Endbenutzer nicht eine Vielzahl von Ladegeräten besitzen (xxxx) sondern benötigt nur eine intelligente Ladestation. %Die anfallenden eventuellen Mehrkosten eines intelligenten Ladesystems im Vergleich zu herkömmlichen Ladestationen, werden hiermit kompensiert. Damit ist eine optimale Ladung der Batterie möglich und somit mit einer Lebensdauerverlängerung zu rechnen.

%Durch Verfolgung des oben vorgestellten Konzept kann der Einsatz einer Vielzahl unterschiedlicher Ladegeräte wegfallen. 
Ähnlich dem Ansatz den die Automobilbrachen zur Zeit verfolgt, können mit der installierten Leistungen auch tragbare Informationsgeräten wie Laptop oder Handy geladen werden. 
%Ebenfalls denkbar ist die Schaffung einer Lösung Berührungsloses Laden von tragbaren Endgeräten (Elektrozahnbürste).

Da der Einsatz von Elektromobilität am Gedanken einer CO2-Neutralen Fortbewegung vorbeigeht wenn zur Ladung des Energiespeichers, Energie aus Kohlekraftwerke genutzt wird. (Hier noch Konkrete Zahlen zum Strommix), ist es zwingend notwendig Regenerative Energiequellen zu nutzen. In unsrem Breitengrad bietet es sich an den Fokus hier auf die Photovoltaik zu legen.\footenote{(Bereits Erfahrungen des Antragstellers vorhanden vgl. Abschnit:xxxx)}

Für den Einsatz in  schützenswerten Umgebungen und/oder mit beschränktem Zugang zum öffentlichen Versorgungsnetz soll der Einsatz von Autarken Ladesäulen forciert werden.

Besonders in der unten genanten Modelllegion bietet sich eine solche Lösung optimal an. 
Argumente wie geringer Bauraum, minimaler Installationsaufwand, und xxx  / Schutz der Umwelt unterstreichen diesen Ansatz noch einmal.

Projektpartner DMOS?

Außerdem lassen sich somit Austauschstationen realisieren, die lediglich einen Batterytypen parat halten müssen. (vgl.Ladetechnik)

\section{Nutzen und Einsatzpotential}
% Warum Boote und Spreewald ?
% erst klären warum Spreewald und dann auf Kahn kommens

Ziel des Vorhabens ist es anhand einer Modellregion den effizienten Einsatz von Elektromobilität mit Messungen und Zahlen zu belegen. Hierbei soll nicht nur die Machbarkeit im Vordergrund stehen sondern auch wirtschaftliche Interessen gewahrt bleiben.

\subsection{Modellregion Spreewald}

%Hierfür soll das Biosphärenreservat Spreewald als Modellregion dienen

Als Modellregion zum Einsatz und testen eines solchen ganzheitlichen Elektromobilitätskonzeptes bietet sich hervorragend das Biosphärenreservat Spreewald an. 
Diese Region ist charakterisiert durch eine einzigartige Flora und Fauna.%, welche schützenswert ist.

Da die Erdhaltbarkeit der Einzigartigkeit die dem  Spreewald zu Grunde liegt gewahrt werden muss, ist hier ein gesteigertes Interesse zu beobachten. 

Die Ergebnisse die in dieser Region gesammelt werden können, sollen sich auch auf andere Standpunkte in Deutschland übertragen lassen. Maßgeblich möchte der Antragsteller ein nachhaltiges Konzept zum Schutz der Umwelt unter Berücksichtigung wirtschaftlicher Interessen schaffen. 
Durch den Beweis eines nachhaltigen Energiekonzeptes können andere Regionen mit ähnlichen Bestimmungen davon profitieren.

Der Spreewald lockt, durch das befahren vielen hunderter Fliese mit einem Spreewald typischen Spreewaldkahn.Als  problematisch einzustufen ist hier die Verwendung eines Verbrennung- Außenborderdmotors welcher zur Fortbewegung genutzt wird. Erste Studien gemeinsam mit dem Wasser und Bodenverband Oberland Calau haben ergeben, dass der Einsatz von Elektromotoren höchst erstrebenswert und realisierbar ist. Da es derzeit aber noch keine Möglichkeiten gibt die guten Vorsätze umzusetzen sieht der Antragssteller hier ein vermehrten nutzen für das zu entwerfende Batteriesystem.
Da laut oben genannte Definition der Spreewaldkahn in die Kategorie LEVs eingeordnet werden kann, soll in einem ersten Arbeitsschritt ein solches System aufgebaut werden (vgl. Meilensteinplan).
Weitere positive Effekte wie Schonung der Fauna durch geräuscharme Fortbewegung, oder Steigerung der Fahrzeiten durch geeignete Beleuchtungskonzepte sollen hier nur kurz erwähnt sein.

Eine CO2-neutralen Fortbewegung ist aus Sicht des Antragsstellers nicht nur aus Umweltpoltischen Betrachtungen erstrebenswert sondern auch von erhöhtem wirtschaftlichem Interesse. 

Erste Sondierungsgespräche ergaben, dass der Einsatz von Elektromobilität nach Aussagen des Geschäftsführer des Tourismusverbandes eine durchaus positive Wirkung auf die Touristenanzahl habe wird. 

Hier ergibt sich eines Symbiose zwischen wirtschaftlichem nutzen und Umweltgerechten Fortbewegen.


Der Antragsteller sieht diese Projekt auch als Chance  vorherrschende Vorurteile, welche die Elektromobilität zur Zeit noch inne hat durch den Beweis der XXXM achbarkeit abzubauen.
(vgl. auch abschnitt neue Medien)

Als Ziel soll ein ganzheitliches Konzept entwickelt und realisiert werden, welches es dem Besucher gestattet den gesamten Spreewald elektrisch zu erkunden. Mithilfe des obigen Ansatzes sollen als Modellregion verschieden Hotspots eingerichtet werden in dem es Lade-/ sowie Austauschmöglichkeiten gibt.{vgl. Ladetechnik}

Schneller und unkomplizierter Wechsel des Fortbewegungsmittels, aufgrund Landschaftlicher Begebenheiten oder persönlichem Empfinden ist damit problemlos realisierbar. So kann Beispielsweise  der erste Teil mit einem Elektroboot zurückgelegt werden, und der Rest der Strecke mit dem Pedelec.

Weiter positiver Nebeneffekte die die Bearbeitung dieses Themengebiets mit sich bringt ist die Möglichkeit die als wirtschaftlich eher schwachen einzustufende Region durch Steigerung des Tourismusaufkommens positiv zu beeinflussen.

\subsection{ Einordnung der Wasserfahrzeuge im Gebiet der Elektromobilität}
\label{sec:EinordnungDerWasserfahrzeugeImGebietDerElektromobilität}

% Überleitung zum Kahn
Einsatz soll dieses System unter anderem in der bisher nicht im Fokus der Elektromobilität stehenden nautischen Fortbewegung finden.
Hierzu soll zunächst ein Konzept entwickelt und  realisiert werden, welches im Biosphärenreservat Spreewald seinen Einsatz finden soll.

Im Bereich der Forschung und Entwicklung spielt der Einsatz der Elektromobilität im nautisch Umfeld derzeit eine eher untergeordnete Rolle. Das erhöhte Potential im Bezug auf ein multiple einsetzbaren Batteriesystems ist aus Sicht des Antragsteller gegeben. 

Ergebnisse und Erfahrungen auf diesem Gebiet können ohne weiteres auf andere Landschaftsstriche übertragen werden (Seenlandschaft Leipzig etc. Erhöhung der Fahrqualität und Schaffung neuer Potential durch Mitführung eines Energiespeichers (Nachtfahrt, Klimatisierung etc.).

Notwendige Richtlinien zum Umbau eines solchen LEVS sollen in Absprache mit dem Landesamt für Bauen und Verkehr geschehen. 

% Zahlen Bilder
Auf weiter Vorteile wie Wartungsarm, verringerte Betriebskosten und Langlebigkeit soll hier nicht weiter eingegangen werden. 

Der erhöhte Anschaffungspreis des Energiespeichers im vergleiche zur konventionellen Fortbewegung soll sich durch oben genannte Konzepte amortisieren.

Untersucht werden hier zunächst lediglich die Außenborderdmotoren welche in die Kategorie der LEVs eingeordnet werden kann.

Der multiple Einsatz des Batteriesystems sieht aber nicht nur den Einsatz in Elektrokähnen vor.
Das Forschungskonzept umfasst eine ganzheitliches Erschließung der Elektromobilität dieser Region durch den Einsatz von LEVs.

So ist es denkbar das, dass gleiche Batteriesystem sowohl als Traktionsspeicher im Elektrokahn als auch im Elektroroller, Fahrrad, Segway (...) eingesetzt werden kann. Die dafür zu entwickelten Ladestationen ermöglichen durch intelligente Ladealgorithmen eine bestmögliche Behandlung der Batterie. 

Damit kann ein ganzheitliches Konzept zu Integration der Elektromobilität angeboten werden.

Engezusammenarbeit mit dem Landeschiffahrstamz

%---------------------------------
% Reichweiteängste / Toolandschaft
%----------------------------------

\section{Verwertung und Abrechnungssystem (Simulation- und Tool-Landschaft)}
%% Verwertung und Abrechnungsystem

Mithilfe eines modularen Batteriesystems und dem  Wissen der zurückzulegenden Strecke kann eine optimale Anzahl von Batteriestacks bestimmt werden \footenote{bei Endverbrauchern, die die Möglichkeit haben mehre Module aufzunehmen}. Somit ist es Möglich den Verbrauch noch einmal zu reduzieren . %--> co2 ersparniss (Eventuell noch eine Grafik Kausalkette) 

Durch Erstellung eine Software und Nutzung von GPS kann mit speziellen Endgeräten die noch zurücklegbare Stecke berechnet werden.(Auch Apps denkbar)

Reichweitenbemessungen vor Antritt einer Fahrt sorgen für die Möglichkeit immer die bestmögliche Batteriestack-Anzahl zu installieren. Außerdem kann man sich so vorab über mögliche Lade/Austauschstationen informieren und ggf. automatisch Vorreservieren.(vgl. Ladetechnik)) (Nutzung neuer Medien Internetpräsenz/Smartphone Apps/Online-Kalkulator)

Somit sind vorbehalte und "`Reichweiteängst"', welche die Elektromobilität mit sich bringt auf ein Minimum reduzierbar.

Unterstützt werden sollen diese Ansätze durch Ganzfahrzeugsimulationen welche das gesamte Projekt begleiten soll.
 
Um genaue Aussagen treffen zu können, soll es ebenfalls Gegenstand dieses Antrages sein die dafür erforderliche Messtechnik zu generieren.

Mithilfe der gewonnen Ergebnisse sollen geeignet Simulationsmodelle entworfen werden. \footenote{Modellierung soll mit Matlab/Simscape und  VHDL-AMS geschehen.}(Fraunhofer-Institut Dresden(xxxx))

Die Simulation soll das ganze Forschungsvorhaben begleiten, somit kann in einer frühen Entwicklungsphase Aussagen zu mehr oder weniger vielversprechenden Ansätzen getroffen werden. 
 
Aus den Ergebnisse von den gewonnen Messungen in Kombination mit dem Simulationsansatz sollen Aussagen zu Langzeitverhalten und Lebensdauer getroffen werden können.

Nochmalige Steigerung des Kosten-Nutzen-Faktors.

%Abrechnungssystem

Ein wirtschaftlicher Mehrnutzen kann aus dem Forschungskonzept generiert werden:
So ist es denkbar  dem Endbenutzer in Form eines monatlichen Beitrags eine gewissen Batteriekapazität sicherzustellen (Ähnlich heute gängige Handyverträge) Somit ist man nicht mehr Eigentümer der Batterie sonder erwirbt lediglich eine Energiemenge ich Form von Batteriekapazität. Der Endanwender ist somit nicht mehr genötigt große Investition zu leisten und somit eher gewillt zum Erwerb eines LEVs. 

Er erwirbt lediglich das LEV, und nutzt eine Energiequelle in Form eines Akkus (Tankstelleprinzip)

Hierfür soll ein Ausleih- und Abrechnungssystem in Zusammenarbeit mit einem Wirtschaftlichen Partner geplant werden. ( Stadt Lübben?)

Damit ist ein  ganzheitliches Konzept der elektrischen, umweltfreundlichen Fortbewegung denkbar.

Dieses Konzept bietet umfangreiche Chancen potentielle Gäste anzulocken und damit eine touristische Vorreiterrolle einzunehmen. 

Abbau von Ängsten und Vorbehalten gegenüber der Elektromobilität will der Antragsteller mit Hilfe neuer Medien (Internetpräsenz) und öffentlichen Veranstaltungen abbauen. 

\section{Demografischer Wandel und positiver Nutzen für Menschen mit einem Handicap}

Durch den Einsatz eines solchen ganzheitlichen Konzepts ist es möglich auch Bürgern mit Einschränkungen oder Behinderungen entgegen zu kommen.
Wie weithin bekannt ist steht Deutschland ein demografischen Wandel bevor, es ist deshalb sinnvoll frühzeitig auch an solche Problemstellungen zu denken.
Im Zuge dieses Projektes sollen mit Partnern Umbauten an speziellen Fahrändern\footnote{auch Spezial Anwendungen wie Rollstühle oder Roolatoren denkbar} vorgenommen werden. Da gerade dieses Menschen einen erhöhten Mehrnutzen von diesem Konzept haben. (Im Laufe der Bearbeitungszeit geht der Antragsteller davon aus, dass gerade aus diesem Sektor eine erhöhte Nachfragen kommen wird)

Durch einen Zuwachs von Touristen auch aus diesem Segment, kann auch wirtschaftlich von einem erhöhten Interesse ausgegangen werden.

Erweiterung des Aktionsradius für Leute mit einem Handicap.

%%

% Dieser Punkt sollte auch ziehen --> da findet man bvestimmt Partner
% Bericht von Prof.Lauckner zu Problemen und Kosten eines solschen Fahrzeuges (Internet Öfeentlcihkeit


Ein vergleichbarer Ansatz mit einem modular aufgebauten und Fahrzeugübergreifenden Akkumulatorsystem konnte am Markt nicht identifiziert werden. Bisherige Lösungen sind jeweils auf ein Fahrzeug zugeschnitten. 


\section{Phrasen und abschließende Sätze}
% Abschließender Satz --> Appel an Elektomobilität
Umweltpolitische Messungen was kann das System bringen, Möglichkeiten und grenzen der Elektromobilität im Bereich der LEVs
Schlüsse auf andere Ähnlich gelagerten Gebiete.

Sowohl Umweltpolitische als auch Energiepolitische Betrachtungen dieses Themas machen den Einsatz von Elektromobilität zwingend notwendig. 


Weiterhin sollen Überlegungen getroffen werden den Einsatzzwecke auch auf ortsunveränderliche Geräte auszuweiten(Stützspeicher).

\begin{itemize}
	\item Umbau eines Rollers auf das neue System
	\item Umbau eines Pedelecs
	\item Umbau eines Rollstuhls
	\item Umbau eines Elektrokahns
\end{itemize}

\chapter{Stand der Wissenschaft und Technik}
\label{sec:StandDerWissenschaftUndTechnik}


\section{Light Electrical Vehicle (LEV)}
\label{sec:LightElectricalVehicleLEV}


\section{Ladeinfastruktur}
\label{sec:Ladeinfastruktur}


\subsection{Autarke Ladesäulen}
\label{sec:AutarkeLadesäulen}


\subsection{Möglichkeiten der Aufladung}
\label{sec:MöglichkeitenDerAufladung}


\section{Solarmodule}
\label{sec:Solarmodule}


\section{Batteriemodule}
\label{sec:Batteriemodule}


Aus dieser Aufgabenstellung ergeben sich folgende Anforderungen an das modulare
\begin{itemize}

\item \begin{itemize}Allgemeine Anforderungen
			\item Spannung
			\item Kapazität
			\item Akkumulatortyp
			\item Sicherheit
			\item Schnittstellen
			\item Informationen
			\item Kommunikation
			\end{itemize}
\item \begin{itemize}Konstruktive Anforderungen
			\item Form
			\item  Gewicht
			\item  Baugröße
			\item  Mechanische Festigkeit
			\item  Bedienbarkeit
			\item  Äußeres Erscheinungsbild
			\item  Aufnahmesystem
			\end{itemize}
\end{itemize}





\section{Bisherige Arbeiten des Antragstellers}
\label{sec:BisherigeArbeitenDesAntragstellers}

Einer der Ausgangspunkte für die Entwicklung eines modularen Akkumulatorsystems ist das Projekt "`Untersuchung über die Einsatzmöglichkeiten von Elektromobilität in Spreewaldkähnen"'. In diesem Projekt werden die Einsatzmöglichkeiten der
elektrischen Mobilität in Wasserfahrzeugen am Beispiel der Spreewaldkähne im
Biosphärenreservat Spreewald untersucht.

% CO2-Thema obwohl ich nicht weiß ob man das erwähnen sollte
% Saxmobility
% optimierte Fahrzeuglängsführung


\section{Antrag Stettler}



\chapter{Ausf�hrliche Beschreibung des Arbeitsplanes}
\label{sec:Ausf�hrlicheBeschreibungDesArbeitsplanes}


\section{Vorhabenbezogene Ressourcenplanung}
\label{sec:VorhabenbezogeneRessourcenplanung}


\section{Meilensteinplanung}
\label{sec:Meilensteinplanung}

Der geplante L�sungsweg wurde bereits aufgezeichnet und durch einen Meilensteinplan gegliedert.

\begin{itemize}
\item	�bersicht �ber Elektromotoren
\item	Tests mit Booten
\item	Planung Gesamtsystem
\item	Planung der Umr�stung der Wasserfahrzeuge
\item	Batteriesystem
\item	Ladeinfrastruktur
\item	Aufbau Vertrieb und Service
\end{itemize}

\chapter{Verwertungsplan}
\label{sec:Verwertungsplan}


\section{Wirtschafliche Erfolgsasussichten}
\label{sec:WirtschaflicheErfolgsasussichten}

\subsubsection{Verwertungsperspektive Firma BudichPool }
Die Firma Budichpool als unser Kooperationspartner erhofft sich durch die Schaffung des Gesamtsystems zur elektrischen Fortbewegung in Wasserfahrzeugen Aufträge hinsichtlich der technischen Umrüstung der Boote. Daher rührt auch die Bereitschaft zur Zusammenarbeit und der finanziellen Unterstützung. Nach Ablauf des dreijährigen Entwicklungszeitraumes  soll eine Firmenneugründung stehen, die sich nach der Entwicklung des geplanten Batteriesystems mit deren Herstellung und Vermarktung befasst. Dieses Unternehmen soll zusätzlich auch eine Ladeinfrastruktur in Wassersportgebieten sowie einen Verleihservice auf den Markt bringen. 
In Verbindung mit den entsprechenden Herstellern sollen auch die Einsatzmöglichkeiten des Batteriesystems in anderen Elektrofahrzeugen sowie möglichen weiteren Anwendungen ausgelotet werden.  

Es ist noch festzuhalten, dass dieses Projekt für alle Beteiligten ein gewisses technisch-wissenschaftliches und wirtschaftliches Risiko birgt. 

\subsubsection{Verwertungsperspektive Tourismusverband Lübben }
\subsubsection{Anreizeffekt}

In der Zusammenarbeit zwischen unserer Hochschule, einem Wirtschaftsunternehmen und anderen teils behördlichen Partnern  sehen wir die optimale Konstillation für die zeitnahe Erreichung der gesteckten Ziele. Da die finanziellen Mittel an den Hochschulen jedoch oft nur eine unbefriedigende Entwicklungsgeschwindigkeit ermöglichen, sind wir uns absolut sicher, mit der Unterstützung durch die Förderungmaßnahme "ELEKTROmobilität: Positionierung der neuen WERtschöpfungskette (ELEKTRO POWER)" die Ziele deutlich schneller und qualitativ hochwertiger zu erreichen um damit auf dem deutschen Markt Fuss zu fassen. Auch unser Projektpartner sieht sich im Stande bei einer derartigen Beihilfe seine eingebrachten Mittel aufzustocken. 
\subsubsection{Biospährenrervat Spreewald}

\subsubsection{Landesamt für Bauen und Verkehr }


\subsection{Warum LEVs Möglichkeiten und Grenzen?}
\label{sec:WarumLEVsMöglichkeitenUndGrenzen}


\subsection{Solarboote Unterstützung in der FahrgastSchiiffahrt?}
\label{sec:SolarbooteUnterstützungInDerFahrgastSchiiffahrt}

\section{Ausgründung einer Firma}
\label{sec:AusgründungEinerFirma}




\section{Wissenschaftliche und/oder technische Erfolgsaussichten}
\label{sec:WissenschaftlicheUndOderTechnischeErfolgsaussichten}

Überprüfung auf Patentwürdigkeit!

\section{Wissenstransfer}
\label{sec:Wissenstransfer}

Ergebnisse werden der öffentlichkeit über ein WIKI bzw. Internetpräs bekannt gegeben.

Öffentlichkeitswirksam sehr hoch
Gut für die Emobilität



\chapter{Zusammenarbeit mit Dritten}
\label{sec:ZusammenarbeitMitDritten}


\begin{enumerate}


\item	Das Biosphärenreservat Spreewald sichert finanzielle, informelle sowie organisatorische Unterstützung zu.
Es soll ein Auftrag für ein Gutachten über die Einsatzmöglichkeiten der Elektromobilität in Spreewaldkähnen an uns erfolgen.

\item	Der Wasser- und Bodenverband Oberland Calau Hat als Pflege- und Instandhaltungseinrichtung für den Spreewald zahlreiche Kähne und professionelles Personal sowie bereits langjährige Erfahrung mit Elektroantrieben.

\item	Das Landesamt für Bauen und Verkehr Brandenburg Möchte mit uns hinsichtlich der Gestaltung der Vorschriften für die technische Umsetzung der Umrüstung der Wasserfahrzeuge zusammenarbeiten und zählt auf unsere wissenschaftliche Kopetenz


\item	Das Autocenter Kamenz, vertreten durch den Geschäftsführer Herrn Böttcher, der ebenfalls jahrzehntelange Erfahrung mit Elektroantrieben in Wasserfahrzeugen hat, unterstützt technisch, durch zahlreiche wertvolle Informationen sowie durch die Bereitstellung eines privaten und geeigneten Sees zu Testzwecken.
\item Schlodarik, Leider Wasser und Bodenverband Calau
\item BudichPool GmbH
	
\end{enumerate}

%\subsection{Kooperationspartner und Arbeitsteilung}
%
%Als Kooperationspartner zum Thema EKahn zeichnet die Firma Budichpool GmbH verantwortlich. Sie ist ein regional im Spreewald ansässiges Unternehmen mit langjähriger Erfahrung im Bereich Metallverarbeitung, Konstruktionstechnik, Planung und PVC-Verarbeitung. Die nötigen Räumlichkeiten, Maschinen und Werkzeuge sowie Personal mit entsprechenden fachlichen Qualifikationen sind vorhanden und für das Projekt eingeplant. Ansprechpartnerist  Herr Burkhard Budich.
%
%Das Unternehmen Budichpool GmbH wurde 1990 gegründet - als GbR  (Burkhard und Frank Budich) und umfasste Folienschweißtechnik, Reifenservice sowie Schwimmbeckenbau. Im Jahr 1994 erfolgte die Gründung einer GmbH (Schwimmbecken- und Freizeitanlagen Budich GmbH).
%Das Unternehmen hat seinen Sitz in 15907 Lübben, an der Feuerwache 2 im Landkreis Dahme Spreewald (LDS). 
%In den vergangenen Jahren wurden zahlreiche Neuentwicklungen im Schwimmbadbereich durchgeführt, wie z. B. Neuentwicklung von Formsteinbecken und Segmentbecken. Letzteres erfolgte mit einer Gebrauchsmustereintragung.  Die Produkte wurden überwiegend in Brandenburg, Sachsen und Berlin vertrieben.
%
%Innovative Aspekte:
%In Zukunft wird sich die Firma mit dem Umbau von Spreewaldkähnen beschäftigen, dies beinhaltet sowohl die Umrüstung von Verbrennungskraftmaschinen auf Elektroantriebe als auch die Entwicklung neuer innovativer Überdachungskonzepte. Damit soll der Einsatz der Spreewaldkähne, welche maßgeblicher für den Tourismus im Spreewald verantwortlich sind, auch bei schlechten Witterungsverhältnissen möglich sein. Es wird davon Ausgegangen, dass der derzeitige Einsatz (Mai-September) um 2-3 Monate erhöht werden kann. Die Firma hat daher größtes Interesse sich an diesem Projekt zu beteiligen um ein Schlüssiges Gesamtkonzept (elektro Kahn + Überdachung) anbieten zu können.



Hallo ich bin ein Test !






%%%----------------------------%%%
%%%  Jetzt geht es zu Ende...  %%%
%%%----------------------------%%%

%%
%%  Bibliographie
%%




%%
%% Stichwortverzeichnis:
%%
%\renewcommand{\indexname}{Stichworte}  % Soll der "Index" anders heißen?
%\printindex                            % Stichwortverzeichnis ausgeben.

\end{document}
%%%%%%%%%%%%%%
%%%  Ende  %%%
%%%%%%%%%%%%%%